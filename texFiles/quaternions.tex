\documentclass[letterpaper]{article}
%------------------------------------------------------------------------------------------
\title{Equations and Derivations for the Quaternion Python Module}
\author{Philippe Pinard}
\date{\today}
%------------------------------------------------------------------------------------------
\usepackage[letterpaper,top=2.5cm,bottom=2.5cm,right=2.5cm,left=2.5cm]{geometry}
\usepackage[english]{babel}
\usepackage[latin1]{inputenc}
%------------------------------------------------------------------------------------------
%\usepackage{fullpage}
\usepackage{graphicx}
\usepackage{subfigure}
\usepackage{multirow}
\usepackage{url}
\usepackage{amsmath}
\usepackage{amsfonts}
\usepackage{stmaryrd}
\usepackage{setspace}
\usepackage{sistyle}
\usepackage[nothing]{todo}
\usepackage[version=3]{mhchem}
\usepackage{fancyhdr}
\usepackage{paralist}
%------------------------------------------------------------------------------------------
\usepackage[
	pdftitle={Equations and Derivations for the Quaternion Python Module},
	pdfsubject={Programming notes},
	pdfkeywords ={Quaternions, Rotation, Python, Derivations},
	pdfauthor={Philippe Pinard},
	colorlinks=true,
	linkcolor=blue,
	pdfborder=0 0 0,
	pdfhighlight=/I,
	pdfpagelabels]{hyperref}
%------------------------------------------------------------------------------------------

\begin{document}
%------------------------------------------------------------------------------------------
	\pagestyle{fancy}
	\fancyhf{}
	\setlength{\headheight}{15pt}
	\setlength{\headsep}{10pt}
	\lhead{Equations and Derivations for the Quaternion Python Module}
	\rhead{\today}
	\lfoot{\emph{Prepared by Philippe Pinard}}
	\rfoot{\thepage}
%------------------------------------------------------------------------------------------
	\newcommand{\dev}{\ensuremath{\ \mathrm{d}}}
	\renewcommand{\labelitemii}{$\diamond$}
	\newcommand{\celsius}{^\circ C}
	\renewcommand{\tablename}{Table}
	\renewcommand{\figurename}{Figure}
	\newcommand{\conj}[1]{#1^\ast}
	\newcommand{\trans}[1]{#1^\mathrm{T}}
	\newcommand{\vx}{\hat{x}}
	\newcommand{\vy}{\hat{y}}
	\newcommand{\vz}{\hat{z}}
	\newcommand{\trace}[1]{\mathrm{Tr}(#1)}
	\newcommand{\quaternion}[2]{\llbracket #1, #2 \rrbracket}
	\newcommand{\quaternionL}[1]{\mathcal{#1}}
	\newcommand{\norm}[1]{\left\|#1\right\|}
	\newcommand{\im}{\mathit{i}}
	\newcommand{\direction}[1]{\left[ #1 \right]}
	\newcommand{\vecc}[3]{\left(#1,#2,#3\right)}
	
	
	\section{Definitions and Notations}
	
	\subsection{Quaternions}
	\begin{itemize}
		\item Reference(s)
			\begin{itemize}
				\item \cite{Altmann1986}
			\end{itemize}
		\item Definition(s)
			\begin{itemize}
				\item $\quaternionL{A} = \quaternion{a}{\vec{A}}$
				\item $a$: a real scalar number
				\item $\vec{A}$: a vector in $\mathbb{R}^3$
				\item Quaternion algebra obeys to all the usual arithmetical laws except that the multiplication is non-commutative
			\end{itemize}
		\item Notation(s)
			\begin{itemize}
				\item $\quaternionL{A}$: a quaternion
				\item $\quaternion{a}{\vec{A}}$: a quaternion composed of a scalar $a$ and a vector $\vec{A}$
				\item $\vec{A} = \vecc{A_x}{A_y}{A_z}$
				\item $\quaternion{a}{\vecc{A_x}{A_y}{A_z}}$: other representation of the scalar/vector components of the quaternion
				\item $\norm{\quaternionL{A}}$: norm of the quaternion
				\item $\conj{\quaternionL{A}}$: conjugate of the quaternion
				\item $\quaternionL{A}^{-1}$: inverse of the quaternion
			\end{itemize}
		\item Advantage(s)
			\begin{itemize}
				\item Easy to determine geometrically as $R\left(\phi\vec{n}\right)$
				\item They are the only parameters for which the group multiplication rule can be given in closed form
				\item They behave correctly near the identity whereas the euler angles become undetermined for $\beta=0$ and $\beta=\pi$
				\item They uniquely determine rotation poles with the convention of the positive hemi-sphere
					\begin{itemize}
						\item $z>0$
						\item if $z=0$, $x>0$
						\item if $z=0$ and $x=0$, $y>0$
					\end{itemize}
				\item Keep track of $2\pi$ turns introduced on multiplying rotations
			\end{itemize}
	\end{itemize}
	
	\subsection{Euler Angles}
	\label{sec:eulerangles}
	\begin{itemize}
		\item Reference(s)
			\begin{itemize}
				\item \cite{Rollett2008}
			\end{itemize}
		\item Definition(s)
			\begin{itemize}
				\item From Bunge
				\item Rotation performed from $\theta_1 \rightarrow \theta_2 \rightarrow \theta_3$ where:
					\begin{enumerate}
						\item $\theta_1$ is a counter-clockwise rotation around the $\vec{z}$ axis $\direction{001}$: $R\left(\theta_1\vec{z}\right)$
						\item $\theta_2$ is a counter-clockwise rotation around the $\vec{x}$ axis $\direction{100}$: $R\left(\theta_2\vec{x}\right)$
						\item $\theta_3$ is a counter-clockwise rotation around the $\vec{z}$ axis $\direction{001}$: $R\left(\theta_3\vec{z}\right)$
					\end{enumerate}
				\item Limits
					\begin{itemize}
						\item $-\pi < \theta_1 \leq \pi$
						\item $0 \leq \theta_2 \leq \pi$
						\item $-\pi < \theta_3 \leq \pi$
					\end{itemize}
			\end{itemize}
	\end{itemize}

%------------------------------------------------------------------------------------------

\newpage
	\section{Algebra}
	
	\subsection{Product}
	\begin{itemize}
		\item Reference(s)
			\begin{itemize}
				\item \cite{Altmann1986}
			\end{itemize}
		\item Properties
			\begin{itemize}
				\item Non-commutative
					\begin{itemize}
						\item $\quaternionL{A}\quaternionL{B} \neq \quaternionL{B}\quaternionL{A}$
					\end{itemize}
				\item Associative
					\begin{itemize}
						\item $\left(\quaternionL{AB}\right)\quaternionL{C}=\quaternionL{A}\left(\quaternionL{BC}\right)$
					\end{itemize}
				\item Scalar product
					\begin{itemize}
						\item $a\quaternion{b}{\vec{B}} = \quaternion{a}{\vec{0}}\quaternion{b}{\vec{B}}=\quaternion{ab}{a\vec{B}}$
					\end{itemize}
				\item Quaternion product
					\begin{itemize}
						\item $\quaternionL{C} = \quaternionL{AB} = \quaternion{a}{\vec{A}}\quaternion{b}{\vec{B}} = \quaternion{ab-\vec{A}\bullet\vec{B}}{a\vec{B}+b\vec{A}+\vec{A}\times\vec{B}}$
							\begin{itemize}
								\item $c = ab-\vec{A}\bullet\vec{B}$
								\item $C_x = aB_x + bA_x + \left(\vec{A}\times\vec{B}\right)_x$
								\item $C_y = aB_y + bA_y + \left(\vec{A}\times\vec{B}\right)_y$
								\item $C_z = aB_z + bA_z + \left(\vec{A}\times\vec{B}\right)_z$
							\end{itemize}
					\end{itemize}
			\end{itemize}
		\item Derivation(s)
			\begin{itemize}
				\item \cite{Altmann1986} or \url{http://www.euclideanspace.com/maths/algebra/realNormedAlgebra/quaternions/arithmetic/index.htm}
			\end{itemize}
	\end{itemize}
	
	\subsection{Division}
	\begin{itemize}
		\item Reference(s)
			\begin{itemize}
				\item \cite{Altmann1986}
			\end{itemize}
		\item Properties
			\begin{itemize}
				\item Scalar division
					\begin{itemize}
						\item $\frac{\quaternion{a}{\vec{A}}}{a} = \frac{1}{a}\quaternion{a}{\vec{A}}$
					\end{itemize}
				\item Quaternion division
					\begin{itemize}
						\item $\frac{\quaternion{a}{\vec{A}}}{\quaternion{b}{\vec{B}}} \equiv \quaternion{a}{\vec{A}}\quaternion{b}{\vec{B}}^{-1}$
						\item We defined this relationship, since $\frac{\quaternionL{A}}{\quaternionL{B}}$ could be equal to $\quaternionL{A}\quaternionL{B}^{-1}$ or $\quaternionL{B}^{-1}\quaternionL{A}$ which doesn't respect the non-commutative rule
						\item $\quaternion{a}{\vec{A}}\quaternion{b}{\vec{B}}^{-1} = \quaternion{ab + \vec{A}\bullet\vec{B}}{b\vec{A} - a\vec{B} - \vec{A}\times\vec{B}}$
					\end{itemize}
			\end{itemize}
		\item Derivation(s)
			\begin{itemize}
				\item \cite{Altmann1986} or \url{http://www.euclideanspace.com/maths/algebra/realNormedAlgebra/quaternions/arithmetic/index.htm}
			\end{itemize}
	\end{itemize}
	
	\subsection{Addition / Substraction}
	\begin{itemize}
		\item Reference(s)
			\begin{itemize}
				\item \cite{Altmann1986}
			\end{itemize}
		\item Properties
			\begin{itemize}
				\item Addition
					\begin{itemize}
						\item $\quaternion{a}{\vec{A}} + \quaternion{b}{\vec{B}} = \quaternion{a + b}{\vec{A} + \vec{B}}$
					\end{itemize}
				\item Subtraction
					\begin{itemize}
						\item \item $\quaternion{a}{\vec{A}} - \quaternion{b}{\vec{B}} = \quaternion{a - b}{\vec{A} - \vec{B}}$
					\end{itemize}
			\end{itemize}
		\item Derivation(s)
			\begin{itemize}
				\item \cite{Altmann1986} or \url{http://www.euclideanspace.com/maths/algebra/realNormedAlgebra/quaternions/arithmetic/index.htm}
			\end{itemize}
	\end{itemize}
	
	\subsection{Conjugate}
	\begin{itemize}
		\item Reference(s)
			\begin{itemize}
				\item \cite{Altmann1986}
			\end{itemize}
		\item Definition(s)
			\begin{itemize}
				\item $\conj{\quaternionL{A}} = \conj{\quaternion{a}{\vec{A}}} = \quaternion{a}{-\vec{A}}$
			\end{itemize}
		\item Derivation(s)
			\begin{itemize}
				\item \cite{Altmann1986} or \url{http://www.euclideanspace.com/maths/algebra/realNormedAlgebra/quaternions/functions/index.htm}
			\end{itemize}
	\end{itemize}
	
	\subsection{Norm}
	\begin{itemize}
		\item Reference(s)
			\begin{itemize}
				\item \cite{Altmann1986} and \cite{Baker2008}
			\end{itemize}
		\item Definition(s)
			\begin{itemize}
				\item $\norm{\quaternionL{A}} = \quaternionL{A}\conj{\quaternionL{A}}$
				\item $\norm{\quaternion{a}{\vec{A}}} = \quaternion{a}{\vec{A}}\conj{\quaternion{a}{\vec{A}}}$
				\item $\norm{\quaternion{a}{\vec{A}}} = \sqrt{a^2 + A_x^2 + A_y^2 + A_z^2}$
			\end{itemize}
		\item Properties
			\begin{itemize}
				\item $\quaternionL{A} = \quaternion{\cos\alpha}{\sin\alpha\vec{n}} = \cos\alpha + \sin\alpha\quaternionL{N}$, $\norm{\vec{n}} = 1$
				\item or $\mathcal{A} = \quaternion{\cos\frac{\phi}{2}}{\sin\frac{\phi}{2}\vec{n}}$ (as in Euler-Rodrigues)
				\item The product of 2 normalized quaternions is itself a normalized quaternion
			\end{itemize}
		\item Derivation(s)
			\begin{itemize}
				\item \cite{Altmann1986} or \url{http://www.euclideanspace.com/maths/algebra/realNormedAlgebra/quaternions/functions/index.htm}
			\end{itemize}
	\end{itemize}
	
	\subsection{Inverse}
	\begin{itemize}
		\item Reference(s)
			\begin{itemize}
				\item \cite{Altmann1986}
			\end{itemize}
		\item Definition(s)
			\begin{itemize}
				\item $\quaternionL{A}^{-1} = \conj{\quaternionL{A}}\norm{\mathcal{A}}^{-2}$
				\item $\quaternionL{C} = \quaternionL{A}\quaternionL{B}^{-1} = \quaternionL{A}\conj{\quaternionL{B}}\norm{\quaternionL{B}}^{-2}$ if $\quaternionL{B} \neq \quaternion{0}{\vec{0}}$
				\item For normalized quaternion, $\quaternionL{A}^{-1} = \conj{\quaternionL{A}}$
			\end{itemize}
		\item Derivation(s)
			\begin{itemize}
				\item \cite{Altmann1986} or \url{http://www.euclideanspace.com/maths/algebra/realNormedAlgebra/quaternions/functions/index.htm}
			\end{itemize}
	\end{itemize}
	
	\subsection{Equality}
	\begin{itemize}
		\item Definition(s)
			\begin{itemize}
				\item Two quaternions ($\quaternionL{A}$ and $\quaternionL{B}$) are equal if and only if
					\begin{itemize}
						\item $a = b$
						\item $A_x = B_x$
						\item $A_y = B_y$
						\item $A_z = B_z$
					\end{itemize}
			\end{itemize}
	\end{itemize}
	
%------------------------------------------------------------------------------------------

\newpage
	\section{Conversion}
	
	\subsection{Axis Angle to Quaternion}
	\begin{itemize}
		\item Reference(s)
			\begin{itemize}
				\item \cite{Baker2008}
			\end{itemize}
		\item Definition(s)
			\begin{itemize}
				\item $\phi$: Rotation angle
				\item $\vec{n}$: Rotation axis
			\end{itemize}
		\item Equation(s)
			\begin{itemize}
				\item $\left(\phi,\vec{n}\right) \rightarrow \quaternion{\cos{\frac{1}{2}\phi}}{\frac{\sin{\frac{1}{2}\phi}}{\norm{\vec{n}}} \vec{n}}$
			\end{itemize}
		\item Derivation(s)
			\begin{itemize}
				\item \url{http://www.euclideanspace.com/maths/geometry/rotations/conversions/angleToQuaternion/index.htm}
			\end{itemize}
	\end{itemize}
	
	\subsection{Matrix to Quaternion}
	\begin{itemize}
		\item Reference(s)
			\begin{itemize}
				\item \cite{Baker2008}
			\end{itemize}
		\item Definition(s)
			\begin{itemize}
				\item $m$: a $3\times 3$ orthogonal matrix (i.e.\ SO3)
					\begin{itemize}
						\item $\mathrm{Det}(m) = 1$
						\item $\trace{m} > 0$
					\end{itemize}
				\item $m_{ij}$: element of matrix $m$ in row $i$ and column $j$
			\end{itemize}
		\item Equation(s)
			\begin{itemize}
				\item $a = \frac{1}{2}\sqrt{1 + m_{00} + m_{11} + m_{22}} = \frac{1}{2}\sqrt{1 + \trace{m}}$
				\item $A_x = \frac{m_{21} - m_{12}}{4a}$
				\item $A_y = \frac{m_{02} - m_{20}}{4a}$
				\item $A_z = \frac{m_{10} - m_{01}}{4a}$
			\end{itemize}
		\item Derivation(s)
			\begin{itemize}
				\item \url{http://www.euclideanspace.com/maths/geometry/rotations/conversions/matrixToQuaternion/index.htm}
			\end{itemize}
	\end{itemize}
	
	\subsection{Euler Angles to Quaternion}
	\label{sec:eulerquaternion}
	\begin{itemize}
		\item Reference(s)
			\begin{itemize}
				\item Adaptation from \cite{Baker2008}, \cite{Rollett2008} and \cite{Wikipedia2008e}
			\end{itemize}
		\item Definition(s)
			\begin{itemize}
				\item $\theta_1, \theta_2, \theta_3$: see \ref{sec:eulerangles}
				\item $c_i \equiv \cos{\left(\frac{1}{2}\theta_i\right)}$ and $s_i \equiv \sin{\left(\frac{1}{2}\theta_i\right)}$
			\end{itemize}
		\item Equation(s)
			\begin{itemize}
				\item $a = c_1c_2c_3 - s_1c_2s_3$
				\item $A_x = c_1s_2c_3 + s_1s_2s_3$
				\item $A_y = c_1s_2s_3 - s_1s_2c_3$
				\item $A_z = c_1c_2s_3 + s_1c_2c_3$
			\end{itemize}
		\item Derivation(s)
			\begin{itemize}
				\item Three rotations
					\begin{enumerate}
						\item $Q_1 = \quaternion{c_1}{\vecc{0}{0}{s_1}}$
						\item $Q_2 = \quaternion{c_2}{\vecc{s_2}{0}{0}}$
						\item $Q_3 = \quaternion{c_3}{\vecc{0}{0}{s_3}}$
					\end{enumerate}
				\item Overall rotation
					\begin{itemize}
						\item $Q_T = Q_3 Q_2 Q_1 = \left(Q_3 Q_2\right) Q_1 = Q_3 \left(Q_2 Q_1\right)$
					\end{itemize}
			\end{itemize}
	\end{itemize}
	
	\subsection{Quaternion to Axis Angle}
	\begin{itemize}
		\item Reference(s)
			\begin{itemize}
				\item \cite{Baker2008}
			\end{itemize}
		\item Equation(s)
			\begin{itemize}
				\item $\norm{\quaternion{a}{A}} = 1$ (the quaternion has to be normalized)
				\item $d = \sqrt{1 - a^2}$
				\item $\phi = 2\arccos{a}$
				\item If $d \neq 0$: $\vec{n} = \frac{1}{d}\vec{A}$
				\item If $d \neq 0$: $\vec{n} = \vec{A}$
			\end{itemize}
		\item Derivation(s)
			\begin{itemize}
				\item \url{http://www.euclideanspace.com/maths/geometry/rotations/conversions/quaternionToAngle/index.htm}
			\end{itemize}
	\end{itemize}
	
	\subsection{Quaternion to Matrix}
	\label{sec:quaternionmatrix}
	\begin{itemize}
		\item Reference(s)
			\begin{itemize}
				\item \cite{Baker2008}
			\end{itemize}
		\item Equation(s)
			\begin{itemize}
				\item $\norm{\quaternion{a}{A}} = 1$ (the quaternion has to be normalized)
				\item $m = \left(\begin{array}{ccc}
												 1 - 2A_y^2 - 2A_z^2		& 2A_xA_y - 2A_za 		& 2A_xA_z + 2A_ya 		\\
												 2A_xA_y + 2A_za 				& 1 - 2A_x^2 - 2A_z^2 & 2A_yA_z - 2A_xa 		\\
												 2A_xA_z - 2A_ya 				& 2A_yA_z + 2A_xa 		& 1 - 2A_x^2 - 2A_y^2 
												\end{array} \right)$
			\end{itemize}
		\item Derivation(s)
			\begin{itemize}
				\item \url{http://www.euclideanspace.com/maths/geometry/rotations/conversions/quaternionToMatrix/index.htm}
			\end{itemize}
	\end{itemize}
	
	\subsection{Quaternion to Euler Angles}
	\begin{itemize}
		\item Reference(s)
			\begin{itemize}
				\item Adaptation from \cite{Baker2008}, \cite{Rollett2008} and \cite{Wikipedia2008e}
			\end{itemize}
		\item Definition(s)
			\begin{itemize}
				\item $\theta_1, \theta_2, \theta_3$: see \ref{sec:eulerangles}
				\item $c_i \equiv \cos{\left(\frac{1}{2}\theta_i\right)}$ and $s_i \equiv \sin{\left(\frac{1}{2}\theta_i\right)}$
			\end{itemize}
		\item Equation(s)
			\begin{itemize}
				\item $\norm{\quaternion{a}{A}} = 1$ (the quaternion has to be normalized)
				\item Three cases
					\begin{enumerate}
						\item $\theta_2 = 0$ ($A_x = A_y = 0$)
							\begin{itemize}
								\item $\theta_1 = 2\arctan{\left(\frac{A_z}{a}\right)}$
								\item $\theta_2 = 0$
								\item $\theta_3 = 0$
							\end{itemize}
						\item $\theta_2 = \pi$ ($A_x^2 + A_y^2 = 1$)
							\begin{itemize}
								\item $\theta_1 = 2\arctan{\left(\frac{A_y}{A_x}\right)}$
								\item $\theta_2 = \pi$
								\item $\theta_3 = 0$
							\end{itemize}
						\item $0 < \theta_2 < \pi$
							\begin{itemize}
								\item $\theta_1 = \arctan{\left(\frac{A_xA_z - A_ya}{A_yA_z + A_xa}\right)}$
								\item $\theta_2 = \arccos{\left(1 - 2A_x^2 - A_y^2\right)}$
								\item $\theta_3 = \arctan{\left(\frac{A_xA_z + A_ya}{A_xa - A_yA_z}\right)}$
							\end{itemize}
					\end{enumerate}
			\end{itemize}
		\item Derivation(s)
			\begin{itemize}
				\item From the Euler to Matrix we get
					\begin{itemize}
						\item $m_\text{euler} = \left(\begin{array}{ccc}
																						c_1c_3 - s_1c_2s_3 			& -s_1c_3 - c_1c_2s_3 			& s_2s_3			\\
																						s_1c_2c_3 + c_1s_3 			& c_1c_2c_3 - s_1s_3				& -s_2c_3			\\
																						s_1s_2									& c_1s_2										& c_2					
																					\end{array} \right)$
						\item We can get the euler angles from these relationships
							\begin{itemize}
								\item $\tan\theta_1 = \frac{m_{20}}{m_{21}} = \frac{s_1}{c_1} = \frac{s_1s_2}{c_1s_2}$
								\item $\cos\theta_2 = m_{22}$
								\item $\tan\theta_3 = -\frac{m_{02}}{m_{12}} = -\frac{s_3}{c_3} = -\frac{s_2s_3}{s_2c_3}$
							\end{itemize}
					\end{itemize}
				\item By comparing with the quaternion matrix (see \ref{sec:quaternionmatrix}), we get replace the $m_{ij}$ by the quaternion coefficients
				\item For the 2 special cases, where the $c_2$ is not defined
					\begin{enumerate}
						\item $\theta_2 = 0$
							\begin{itemize}
								\item This condition happens when
									\begin{itemize}
										\item $\arccos{0} = 1 = 1 - 2A_x^2 - 2A_y^2$
										\item $A_x^2 = - A_y^2$
										\item $A_x = A_y = 0$
									\end{itemize}
								\item $\theta_2 = 0$ implies that $\cos{\left(\frac{1}{2} 0\right)} = c_2 = 1$ and $\sin{\left(\frac{1}{2} 0\right)} = s_2 = 0$
								\item From the Euler to quaternion conversion (see \ref{sec:eulerquaternion})
									\begin{itemize}
										\item $a = c_1c_2c_3 - s_1c_2s_3 = c_1c_3 - s_1s_3$
										\item $A_x = 0$
										\item $A_y = 0$
										\item $A_z = c_1c_2s_3 + s_1c_2c_3 = c_1s_3 + s_1c_3$
									\end{itemize}
								\item Using the trigonometry identities:
									\begin{itemize}
										\item $\cos{\left(A+B\right)} = \cos{A}\cos{B} - \sin{A}\sin{B}$
										\item $\sin{\left(A+B\right)} = \sin{A}\cos{B} + \cos{A}\sin{B}$
									\end{itemize}
								\item We obtained
									\begin{itemize}
										\item $a = \cos{\left(1+3\right)}$
										\item $A_z = \sin{\left(1+3\right)}$
									\end{itemize}
								\item By dividing $A_z$ by $a$:
									\begin{itemize}
										\item $\frac{A_z}{a} = \frac{\sin{\left(1+3\right)}}{\cos{\left(1+3\right)}} = \tan{\left(1+3\right)}$
										\item $\frac{\theta_1}{2} + \frac{\theta_3}{2} = \arctan{\left(\frac{A_z}{a}\right)}$
										\item $\theta_1 + \theta_3 = 2\arctan{\left(\frac{A_z}{a}\right)}$
									\end{itemize}
								\item Since ``only ($\theta_1 + \theta_3$) is uniquely defined (not the individual values)''\cite{Wikipedia2008e} when $\theta_2 = 0$
									\begin{itemize}
										\item $\theta_1 = 2\arctan{\left(\frac{A_z}{a}\right)}$
										\item $\theta_2 = 0$ (from the special case)
										\item $\theta_3 = 0$
									\end{itemize}
							\end{itemize}
						\item $\theta_2 = \pi$
							\begin{itemize}
								\item This condition happens when
									\begin{itemize}
										\item $\arccos{\pi} = -1 = 1 - 2A_x^2 - 2A_y^2$
										\item $A_x^2 + A_y^2 = 1$
										\item Since the quaternion is normalized ($a^2 + A_x^2 + A_y^2 + A_z^2 = 1$), $a = A_z = 0$
									\end{itemize}
								\item $\theta_2 = \pi$ implies that $\cos{\left(\frac{1}{2} \pi\right)} = c_2 = 0$ and $\sin{\left(\frac{1}{2} \pi\right)} = s_2 = 1$
								\item From the Euler to quaternion conversion (see \ref{sec:eulerquaternion})
									\begin{itemize}
										\item $a = 0$
										\item $A_x = c_1s_2c_3 + s_1s_2s_3 = c_1c_3 + s_1s_3$
										\item $A_y = c_1s_2s_3 - s_1s_2c_3 = c_1s_3 - s_1c_3$
										\item $A_z = 0$
									\end{itemize}
								\item Using the trigonometry identities:
									\begin{itemize}
										\item $\cos{\left(A-B\right)} = \cos{A}\cos{B} + \sin{A}\sin{B}$
										\item $\sin{\left(A-B\right)} = \sin{A}\cos{B} - \cos{A}\sin{B}$
									\end{itemize}
								\item We obtained
									\begin{itemize}
										\item $A_x = \cos{\left(1-3\right)}$
										\item $A_y = -\sin{\left(1-3\right)}$
									\end{itemize}
								\item By dividing $A_y$ by $A_x$:
									\begin{itemize}
										\item $\frac{A_y}{A_x} = \frac{-\sin{\left(1-3\right)}}{\cos{\left(1-3\right)}} = -\tan{\left(1-3\right)}$
										\item $\frac{\theta_1}{2} - \frac{\theta_3}{2} = \arctan{\left(\frac{-A_y}{A_x}\right)}$
										\item $\theta_1 - \theta_3 = 2\arctan{\left(\frac{-A_y}{A_x}\right)}$
									\end{itemize}
								\item Since ``only ($\theta_1 - \theta_3$) is uniquely defined (not the individual values)''\cite{Wikipedia2008e} when $\theta_2 = \pi$
									\begin{itemize}
										\item $\theta_1 = 2\arctan{\left(\frac{-A_y}{A_x}\right)}$
										\item $\theta_2 = \pi$ (from the special case)
										\item $\theta_3 = 0$
									\end{itemize}
							\end{itemize}
					\end{enumerate}
			\end{itemize}
	\end{itemize}
	
	\bibliographystyle{plain}
	\bibliography{bibliography}

\end{document}